An emergency evacuation is the time-critical migration of people from a dangerous situation to a position of safety. Disasters such as fires, floods, hurricanes and nuclear hazards can call for evacuations of potentially large regions, and studies have shown that an evacuation of greater than 1000 people occurs, on average, every two to three weeks in the USA \cite{SandiaNationalLaboratories2005IdentificationEvacuations,Murray-Tuite2013EvacuationPractice}. As urban environments become increasingly densely populated, regional infrastructure is often found ill-equipped to handle the stress of an evacuation scenario \cite{Pitt2008TheFloods}, and in recent decades eyes have increasingly turned towards the factors that play into evacuation issues \cite{Dow2002EmergingCarolina} such as households evacuating with more cars than necessary, and the region evacuating too quickly leading to crippling congestion on the road network. If we can understand the factors that are at play in an evacuation, authorities and local governments will be able to design schemes for more efficient and successful egress. Practical evacuation drills on a regional scale are expensive, pose ethical issues and are often simply impossible, for this reason researchers have turned to computational simulation. 

\section{Agent-Based Modelling}
Computer simulation of evacuation scenarios historically used top-down modelling techniques. A top-down technique aims to characterise the macroscopic outcome of a phenomenon using general terms. For example, evacuation systems have previously been modelled using fluid dynamics, drawing parallels between an individual in a crowd with a molecule in a wave \cite{Thompson1995APopulations}.
In contrast, a bottom-up model aims to characterise the individual components of a system on a microscopic basis, modelling their actions and allowing their interactions to naturally produce an \textit{emergent} phenomenon. 

Microsimulation is more computationally taxing than macrosimulation, demanding processing time for each agent in the system; however, bottom-up approaches are able to account for \textit{interactions between agents}, a concept that top-down models are unable to replicate. Modelling scenarios by defining the behaviour of the individual agents is referred to as agent-based simulation or agent-based modelling (ABM). For our purposes we shall use the term ABM to describe such a system.

Microscopic simulation of road networks started becoming computationally feasible in the 1990s, beginning a surge of research that attempted, and succeed, to accurately model road traffic \cite{Nagel1992ATraffic}. As computing power increased further, ABM research accelerated rapidly in the 21st century \cite{Bonabeau2002Agent-basedSystems.,Teodorovic2003TransportApproach}.

\section{Objective}
This report describes the implementation of a robust simulation for\textit{in silico} evacuation experimentation, and

Initially, the paper will overview the general design elements of agent-based modelling, with further explanation given to its implementation in an evacuation context. Existing evacuation models will be considered, and the state of modern ABM systems understood before investigating how researchers have previously leveraged ABM to engineer new strategies for evacuation. The paper will analyse prior research, and speculate on potential advances for additional study. 